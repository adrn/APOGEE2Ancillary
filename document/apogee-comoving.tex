\documentclass[11pt]{article}
\usepackage{geometry}
\usepackage{graphicx}
\usepackage{titling}
\usepackage{amsmath,amssymb}
\usepackage{authblk}

\setlength{\droptitle}{-10em}

\title{Chemical abundances and radial velocities for candidate co-moving stars}

\author{
  Semyeong Oh (Co-PI, \texttt{semyeong@astro.princeton.edu})
  \and
  Adrian M. Price-Whelan (Co-PI, \texttt{adrn@astro.princeton.edu})
}

\date{}

\begin{document}
\maketitle

\section*{Co-Is}

\begin{itemize}
    \item Melissa Ness, Max Planck Institute for Astronomy, Germany
    \item David W. Hogg, New York University, USA
    \item Hans-Walter Rix, Max Planck Institute for Astronomy, Germany
\end{itemize}

\section*{External Collaborators: Requests for External Status}

Adrian M. Price-Whelan: An EC request will be submitted with this proposal.

Adrian M. Price-Whelan has expertise in statistical inference and dynamical
modeling of the Milky Way.

He is an SDSS-III architect for his contributions to the
science archive server, data management software, and observation management
software used by the SDSS observers.

\section{Scientific and technical justification}

\subsection{Summary}

Wide binaries as dynamical tracers

Wide binaries as test-beds for stellar models

Sample 1: Pairs with a red giant member - observe all of these

Sample 2: All other pairs

\end{document}
