\documentclass[11pt]{article}
\usepackage{geometry}
\usepackage{graphicx}
\usepackage{titling}
\usepackage{amsmath,amssymb}
\usepackage{authblk}
\usepackage{longtable}
\usepackage[T1]{fontenc}

\setlength{\droptitle}{-10em}

\title{Chemical abundances and radial velocities for candidate co-moving stars}

\author{
  Semyeong Oh (Co-PI, \texttt{semyeong@astro.princeton.edu})
  \and
  Adrian M. Price-Whelan (Co-PI, \texttt{adrn@astro.princeton.edu})
}

\date{}

\begin{document}
\maketitle

\section*{Co-Is}

\begin{itemize}
    \item Melissa Ness, Max Planck Institute for Astronomy, Germany \item David W. Hogg, New York University, USA
    \item Hans-Walter Rix, Max Planck Institute for Astronomy, Germany
\end{itemize}

\section*{External Collaborators: Requests for External Status}

Semyeng Oh, Adrian M. Price-Whelan: EC requests will be submitted with this proposal.

Semyeong Oh:

Adrian M. Price-Whelan has expertise in statistical inference and dynamical
modeling of the Milky Way.

He is an SDSS-III architect for his contributions to the
science archive server, data management software, and observation management
software used by the SDSS observers.

\section{Scientific and technical justification}

\subsection{Summary}

Our main science goal is to validate and obtain detailed chemical abundances for
a sample of high-confidence, candidate co-moving stars selected using parallaxes
and proper motions from the Tycho-Gaia Astrometric Solution (TGAS, a component
of the first data release of the Gaia mission).
We will use radial velocity measurements of each star in a candidate pair to
identify true co-moving pairs, which span a large range of physical separations
($\approx$0.01--10 pc).

- Emphasize the support provided to APOGEE - the confirmed wide separation pairs
  provide tests relevant to the APOGEE pipeline and results. (test abundances,
  age indicators, prospects of chemical tagging)

... tracers of star-formation processes, are
sensitive to the tidal field and granularity of the Galactic mass distribution,
and will be valuable targets for anchoring stellar models across the HR diagram.

Wide binaries as dynamical tracers

Wide binaries as test-beds for stellar models

\section{Target Information}

We propose to observe 137 stars with $H>7$~mag within existing fields
from two categories:
\begin{itemize}
  \item 65 pairs (130 stars) for which both of the member stars falls
  within the existing fields
  \item 7 additional stars for which the other star in the pairs is aleady
  observed in APOGEE DR12
\end{itemize}
%
There are 21 pairs in the first category whose angular separation is smaller
than 72~arcseconds, the fiber collision threshold. However, given the
brightnesso f these stars, they should be able to reach signal-to-noise ratio
per resolution element over 100 with a single visit.

Mention false-positive rate constraint from RAVE


\begin{longtable}{cccccc}
\caption{137 Targets for Type-I proposal}\label{tab:sample}\\
\hline\hline
RA & DEC & 2MASS ID & H & Pair ID & Field Name\\
 &  &  & mag &  & \\
\hline
287.44730 & 47.94488 & 19094734+4756415 & 7.588 & 552 & K06 078+16 \\
288.83522 & 49.71755 & 19152044+4943030 & 8.508 & 552 & K01 082+17 \\
282.48282 & 0.53724 & 18495587+0032141 & 8.180 & 589 & 034+00 \\
282.87363 & 1.40820 & 18512967+0124295 & 7.809 & 589 & 034+00 \\
64.80293 & 23.86828 & 04191269+2352058 & 9.197 & 808 & K2 C4 172-20 \\
68.09083 & 25.18549 & 04322179+2511078 & 7.104 & 808 & TAUL1521 \\
5.37130 & 68.60564 & 00212913+6836204 & 8.962 & 831 & 120+06 \\
4.70452 & 69.54114 & 00184909+6932282 & 9.880 & 831 & 120+06 \\
333.92394 & 41.79393 & 22154174+4147382 & 9.051 & 863 & 094-12 \\
334.32668 & 43.10903 & 22171839+4306325 & 9.622 & 863 & 094-12 \\
293.31591 & 47.88778 & 19331581+4753159 & 9.573 & 956 & K05 080+14 \\
292.97808 & 47.65320 & 19315472+4739114 & 8.922 & 956 & K05 080+14 \\
165.54212 & 9.89520 & 11021011+0953427 & 9.119 & 1035 & 240+60 \\
165.54535 & 9.89518 & 11021088+0953426 & 8.436 & 1035 & 240+60 \\
294.53851 & 44.15990 & 19380923+4409356 & 8.814 & 1105 & K15 077+10 \\
294.53719 & 44.15380 & 19380891+4409137 & 8.777 & 1105 & K15 077+10 \\
295.04594 & 44.75923 & 19401103+4445333 & 7.741 & 1142 & K10 079+12 \\
299.84244 & 44.58014 & 19592217+4434485 & 8.094 & 1142 & K14 080+08 \\
141.49463 & 48.26897 & 09255870+4816083 & 8.972 & 1161 & ORPHAN-5 \\
143.58371 & 48.21605 & 09342009+4812578 & 9.290 & 1161 & ORPHAN-5 \\
25.22783 & 32.07539 & 01405465+3204314 & 8.686 & 1195 & 135-30 \\
24.85952 & 33.14502 & 01392625+3308423 & 8.074 & 1195 & 135-30 \\
331.50097 & 41.59279 & 22060023+4135340 & 9.580 & 1283 & 094-12 \\
331.49989 & 41.59185 & 22055997+4135306 & 9.588 & 1283 & 094-12 \\
287.32984 & 39.20100 & 19091919+3912037 & 7.391 & 1317 & K18 070+14 \\
287.38206 & 39.18092 & 19093170+3910515 & 8.239 & 1317 & K18 070+14 \\
334.61111 & 49.09115 & 22182666+4905282 & 10.829 & 1352 & 100-06 \\
334.40163 & 49.65119 & 22173638+4939042 & 8.819 & 1352 & 100-06 \\
216.69558 & 50.76833 & 14264694+5046060 & 9.042 & 1367 & 090+60 \\
288.43549 & 47.75517 & 19134451+4745186 & 9.010 & 1417 & K06 078+16 \\
288.41248 & 47.41761 & 19133899+4725035 & 9.438 & 1417 & K06 078+16 \\
24.47087 & 73.70753 & 01375298+7342271 & 8.743 & 1450 & 125+12 \\
18.54890 & 73.60937 & 01141171+7336338 & 10.027 & 1450 & 125+12 \\
7.40019 & 58.14217 & 00293604+5808318 & 9.715 & 1542 & 120-06 \\
5.62270 & 56.36281 & 00222943+5621461 & 9.017 & 1542 & 120-06 \\
85.40995 & -2.25902 & 05413838-0215324 & 7.420 & 1606 & ORIONB-B \\
84.81275 & -2.52712 & 05391506-0231376 & 7.881 & 1606 & ORIONB-B \\
84.05869 & -2.25883 & 05361408-0215318 & 8.459 & 1675 & ORIONOB1AB-A \\
84.64501 & -2.57105 & 05383479-0234158 & 8.380 & 1675 & ORIONOB1AB-A \\
46.14389 & 57.59065 & 03043452+5735262 & 7.679 & 1679 & 139+00 \\
43.53445 & 59.00867 & 02540827+5900312 & 9.473 & 1679 & 139+00 \\
59.95311 & 11.54282 & 03594874+1132341 & 8.467 & 1732 & 180-30 \\
59.95127 & 11.54482 & 03594830+1132412 & 7.549 & 1732 & 180-30 \\
85.38651 & -9.28691 & 05413276-0917128 & 9.324 & 1736 & ORIONA-E \\
45.19293 & 59.65289 & 03004631+5939105 & 7.901 & 1816 & 139+00 \\
45.22410 & 59.66596 & 03005378+5939576 & 7.326 & 1816 & 139+00 \\
226.19370 & 22.41520 & 15044648+2224548 & 6.718 & 1821 & 030+60 \\
2.48313 & 73.48467 & 00095595+7329048 & 8.994 & 1931 & 120+12 \\
2.47943 & 73.48324 & 00095507+7328596 & 8.902 & 1931 & 120+12 \\
352.44315 & 79.90742 & 23294627+7954267 & 8.309 & 1940 & 120+18 \\
351.82688 & 79.79504 & 23271842+7947421 & 7.216 & 1940 & 120+18 \\
286.19371 & 16.42917 &  &  & 1949 & 049+06 \\
284.37367 & 17.35159 & 18572967+1721057 & 8.451 & 1949 & 049+06 \\
330.14783 & 54.17301 & 22003546+5410229 & 8.629 & 1962 & 100+00 \\
330.14831 & 54.17090 & 22003558+5410153 & 8.512 & 1962 & 100+00 \\
266.06857 & -6.52741 & 17441643-0631383 & 7.979 & 1977 & 020+12 \\
266.49529 & -5.75468 & 17455887-0545167 & 9.378 & 1977 & 020+12 \\
341.27451 & 44.81257 & 22450589+4448452 & 8.112 & 1982 & 100-12 \\
341.27332 & 44.81437 & 22450559+4448517 & 8.796 & 1982 & 100-12 \\
98.24364 & -30.86312 & 06325847-3051472 & 8.387 & 1991 & N2243-N \\
329.23498 & 47.15977 & 21565639+4709352 & 9.176 & 2052 & 094-06 \\
327.28275 & 46.79655 & 21490784+4647477 & 10.466 & 2052 & 094-06 \\
295.92588 & 49.44455 & 19434220+4926404 & 7.946 & 2087 & K04 083+13 \\
295.87571 & 48.59270 & 19433017+4835336 & 7.314 & 2087 & K04 083+13 \\
295.00051 & 12.51721 & 19400012+1231019 & 9.177 & 2117 & 049-06 \\
294.85763 & 11.89194 & 19392583+1153310 & 8.540 & 2117 & 049-06 \\
206.53055 & 0.34796 & 13460733+0020528 & 7.758 & 2157 & 330+60 \\
206.32662 & 0.90366 & 13451838+0054132 & 10.634 & 2157 & 330+60 \\
175.82094 & 0.76304 & 11431702+0045468 & 9.108 & 2239 & 270+60 \\
291.07874 & 45.09117 & 19241890+4505282 & 7.713 & 2262 & K11 076+13 \\
320.38854 & 32.88530 & 21213324+3253070 & 8.422 & 2405 & 079-12 \\
320.39355 & 32.88081 & 21213445+3252509 & 9.540 & 2405 & 079-12 \\
85.35621 & 0.96670 & 05412549+0058000 & 8.438 & 2425 & ORIONOB1AB-B \\
86.72955 & 0.96789 & 05465509+0058045 & 7.804 & 2425 & ORIONB-A \\
51.65592 & 62.26128 & 03263741+6215405 & 8.957 & 2505 & 140+06 \\
51.64965 & 62.27708 & 03263591+6216375 & 8.272 & 2505 & 140+06 \\
351.94473 & 52.62956 & 23274672+5237464 & 10.557 & 2535 & 109-08 \\
349.92409 & 53.05291 & 23194177+5303104 & 8.279 & 2535 & 109-08 \\
81.29290 & 1.25870 & 05251029+0115314 & 9.659 & 2595 & ORIONOB1AB-E \\
81.03354 & 2.46302 & 05240804+0227468 & 8.735 & 2595 & ORIONOB1AB-F \\
55.75797 & 64.11413 & 03430191+6406509 & 9.937 & 2619 & 140+06 \\
55.76347 & 64.11158 & 03430324+6406417 & 10.241 & 2619 & 140+06 \\
306.71204 & 39.94384 & 20265088+3956378 & 9.220 & 2716 & 079+00 \\
306.73341 & 39.93830 & 20265601+3956178 & 10.183 & 2716 & 079+00 \\
284.18673 & 42.53470 & 18564481+4232051 & 7.879 & 2837 & K13 071+16 \\
284.18525 & 42.56324 & 18564446+4233477 & 9.280 & 2837 & K13 071+16 \\
30.47984 & 60.45763 & 02015515+6027275 & 9.039 & 2879 & 131+00 \\
30.49735 & 60.46133 & 02015936+6027409 & 7.819 & 2879 & 131+00 \\
287.48157 & 43.85068 & 19095557+4351023 & 9.502 & 2955 & K12 074+15 \\
285.19568 & 41.79322 & 19004696+4147356 & 10.305 & 2955 & K13 071+16 \\
59.68946 & 37.68602 & 03584548+3741096 & 7.841 & 2967 & 160-12 \\
60.96664 & 37.63836 & 04035199+3738181 & 9.324 & 2967 & 160-12 \\
296.78830 & 27.00903 & 19470919+2700325 & 9.124 & 3109 & 064+00 \\
296.78904 & 27.01288 & 19470936+2700464 & 10.152 & 3109 & 064+00 \\
230.03975 & 0.33133 & 15200954+0019528 & 9.607 & 3241 & M5PAL5 \\
228.17263 & 0.70999 & 15124144+0042360 & 9.182 & 3241 & M5PAL5 \\
63.82299 & 61.04117 & 04151751+6102283 & 8.771 & 3346 & 146+08 \\
67.84831 & 61.01069 & 04312358+6100385 & 9.692 & 3346 & 146+08 \\
84.18573 & -6.95370 & 05364456-0657133 & 9.855 & 3479 & ORIONA-C \\
83.77680 & -7.30916 & 05350642-0718330 & 10.204 & 3479 & ORIONA-C \\
280.26069 & 20.16114 & 18410256+2009401 & 10.151 & 3486 & 049+12 \\
280.06667 & 20.50850 & 18401599+2030305 & 8.715 & 3486 & 049+12 \\
121.03517 & 49.30161 & 08040845+4918059 & 8.448 & 3503 & ALPHAPER \\
121.03368 & 49.30240 & 08040808+4918087 & 8.420 & 3503 & ALPHAPER \\
197.61533 & 70.75877 & 13102768+7045315 & 9.598 & 3528 & 120+45 \\
197.68234 & 70.76841 & 13104376+7046061 & 9.228 & 3528 & 120+45 \\
292.15477 & 38.14267 & 19283714+3808335 & 8.864 & 3708 & K21 071+10 \\
292.15576 & 38.13804 & 19283738+3808168 & 9.111 & 3708 & K21 071+10 \\
314.24149 & 37.00053 & 20565796+3700018 & 9.209 & 3742 & 079-06 \\
314.23502 & 37.02323 & 20565640+3701236 & 9.269 & 3742 & 079-06 \\
215.99442 & 56.29905 & 14235864+5617567 & 9.947 & 3792 & GD1-5 \\
215.77210 & 56.46605 & 14230530+5627577 & 9.464 & 3792 & GD1-5 \\
216.35961 & 58.20781 & 14252630+5812281 & 10.669 & 3836 & GD1-5 \\
217.86326 & 58.80584 & 14312719+5848210 & 10.374 & 3836 & GD1-5 \\
332.57548 & 66.13052 & 22101811+6607498 & 9.969 & 3839 & 109+08 \\
332.58715 & 66.11162 & 22102091+6606417 & 10.113 & 3839 & 109+08 \\
54.64420 & 53.86855 & 03383461+5352069 & 7.475 & 3842 & 146+00 \\
54.33817 & 54.28567 & 03372115+5417084 & 8.739 & 3842 & 146+00 \\
64.65663 & 60.83215 & 04183759+6049559 & 8.480 & 3920 & 146+08 \\
64.65682 & 60.83313 & 04183764+6049592 & 8.391 & 3920 & 146+08 \\
67.01184 & 40.18254 & 04280283+4010572 & 8.774 & 3928 & 160-06 \\
65.21842 & 40.10545 & 04205241+4006194 & 7.912 & 3928 & 160-06 \\
294.30470 & 39.45674 & 19371313+3927243 & 10.340 & 3946 & K20 073+09 \\
292.96817 & 38.21574 & 19315237+3812566 & 8.685 & 3946 & K21 071+10 \\
319.77922 & 32.98000 & 21190702+3258480 & 9.563 & 4035 & 079-12 \\
319.63337 & 32.10019 & 21183201+3206007 & 8.737 & 4035 & 079-12 \\
318.45621 & 32.42114 & 21134947+3225160 & 8.074 & 4072 & 079-12 \\
318.45396 & 32.42617 & 21134894+3225342 & 9.933 & 4072 & 079-12 \\
79.92777 & 25.18699 & 05194266+2511132 & 8.805 & 4139 & SGRT-1 \\
79.93198 & 25.18291 & 05194367+2510586 & 8.573 & 4139 & SGRT-1 \\
19.60684 & -12.29880 & 01182564-1217557 & 8.675 & 4181 & 150-75 \\
285.24881 & 16.03242 & 19005971+1601568 & 8.860 & 4275 & 049+06 \\
285.25540 & 16.03453 & 19010129+1602043 & 8.996 & 4275 & 049+06 \\
97.61192 & -31.11808 & 06302686-3107050 & 9.760 & 4291 & N2243-N \\
97.50257 & -31.17448 & 06300060-3110282 & 7.593 & 4291 & N2243-N \\
194.27773 & 42.38495 & 12570666+4223058 & 9.274 & 4465 & 120+75 \\
194.28417 & 42.39118 & 12570821+4223282 & 8.283 & 4465 & 120+75 \\
\hline\hline
\end{longtable}



\end{document}
